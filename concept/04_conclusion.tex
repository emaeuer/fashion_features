\section{Schlussteil} \label{sec:conclusion}

In den beiden vorherigen Abschnitten wurde erläutert, wie die Zielsetzung des Projektes erfüllt werden soll. Dabei wurde der Fashion Product Images Datensatz als Grundlage für das Training gewählt. Es wurden die Kategorien vorgestellt, denen eine Objekt zugeordnet werden kann und wie diese verteilt sind. Die Multi-Label-Klassifizierung soll mithilfe eine CNNs durchgeführt, dass durch Transfer Learning trainierte Schichten verwendet. 

Für eine Evaluierung der Ergebnisse sollen stets Metriken wie Precision und Recall ermittelt werden, um die günstigste Lösungsstrategie zu finden. Infolgedessen kann es im Laufe des Entwicklungsprozesses noch zu Änderungen oder Erweiterungen kommen, die von diesem Konzept abweichen.