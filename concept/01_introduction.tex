\section{Einleitung} \label{sec:intro}

Das Web bietet eine Unmenge an Information, die in den unterschiedlichsten Formaten gespeichert sind. Eine wichtige Maßnahme, um diese Datenflut für den Endnutzer übersichtlich zu gestalten, ist es die verschiedenen Objekte vergleichen zu können. Für einen effektiven Vergleich muss die unstrukturierte Repräsentation der Daten in eine Strukturierte überführt werden.

Ein verbreitetes Verfahren zur Prüfung der Ähnlichkeit von Objekten ist Entity Matching. Dieses, ermöglicht es Entitäten zu vergleichen, die in der Regel in Tabellarischer- oder Textform vorliegen. Allerdings liegen die Daten nicht in allen Anwendungsbereichen in dieser Form vor. Beispielsweise sind Bilder im Fashion-Bereich die primäre Variante der Repräsentation von Produkten. Damit diese effizient miteinander verglichen werden können, müssen zuerst charakteristische Merkmale extrahiert werden.

Das im Folgenden beschriebene Projektkonzept soll sich aus den genannten Gründen damit beschäftigen, wie aus Abbildungen von Kleidung Produkteigenschaften gewonnen werden können. Dies umfasst generell zwei Hauptaufgaben. Zum einen wird ein geeigneter Datensatz benötigt (Abschnitt \ref{sec:data_set}) und zum anderen muss ein Modell erstellt werden, dass durch maschinelles Lernen die Merkmalsextraktion durchführt (Abschnitt \ref{sec:model}). Die Merkmalsextraktion gestaltet sich als ein Problem der "multitask classification" \cite{scikit-learn}.